\documentclass[a4paper,zihao=5,UTF8,fontset=fandol]{phyreport}

%\usetikzlibrary{decorations.pathreplacing}
\usetikzlibrary{decorations.pathmorphing}

\expAcademy{(学院名称)}
\expLesson{(课程名称)}
\expName{(实验名称)}
\expAuthor{(名字)}
\expStuID{(学号)}
\expNum{(组号)}
\expTchr{(教师名字)}
\expDate{}{}{} % these two date are not implemented
\subDate{}{}{}
\expID{(课程编号)}
\expType{(实验类型)}

\begin{document}

\phyExpCover

% 插入预习报告
% \includePDFFirstPage{预习报告文件路径}

\newpage
\fancypage{\fbox}{} % 边框

\section{实验目的}

\smartLongLine
\section{实验原理}

\smartLongLine
\section{实验仪器}

\smartLongLine
\section{实验内容与步骤}

\smartLongLine
\section{数据记录及数据处理}

\begin{table}[h] % h means here, t means top, b means bottom, p means (this) page % you can combine them to form try this then that strategy, like [hb] for first try here, if not then put it at bottom
\begin{tabular}{llllllll} % you need to have exactly one more l than &, for more options, check latex doc
序号 & 1 & 2 & 3 & 4 & 5 & 6 & 7\\
\begin{tabular}[c]{@{}l@{}}数据\\ (?)\end{tabular} & 10 & $\sqrt{\frac{\delta y}{\delta t}}$ & 8 & 7 & 6 & 5 & 4
\end{tabular}
\end{table}

\smartLongLine
\section{结果陈述}

\smartLongLine
\section{思考题}

\endBox

\newpage
\thispagestyle{empty}

\fancypage{ }{}
\section*{原始数据记录表}

\noindent 组号 \rule[-5pt]{2cm}{0.4pt} \ \ 姓名 \rule[-5pt]{2cm}{0.4pt}

% uncomment below graphicx code to paste origin.jpg here
%\begin{figure}[h] % same with that of tables, h means here blablabla
%    \centering
%    \includegraphics[width=1.0\textwidth]{origin.jpg}
%\end{figure}

\end{document}
